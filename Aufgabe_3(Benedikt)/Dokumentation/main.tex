\documentclass[a4paper,10pt,ngerman]{scrartcl}
\usepackage{babel}
\usepackage[T1]{fontenc}
\usepackage[utf8x]{inputenc}
\usepackage[a4paper,margin=2.5cm,footskip=0.5cm]{geometry}

% Die nächsten vier Felder bitte anpassen:
\newcommand{\Aufgabe}{Aufgabe 1: Wandertag} % Aufgabennummer und Aufgabennamen angeben
\newcommand{\TeamId}{12345}                       % Team-ID aus dem PMS angeben
\newcommand{\TeamName}{Team XYZ}                 % Team-Namen angeben
\newcommand{\Namen}{Max Mustermann, Erika Mustermann} % Namen der Bearbeiter/-innen dieser Aufgabe angeben

% Kopf- und Fußzeilen
\usepackage{scrlayer-scrpage, lastpage}
\setkomafont{pageheadfoot}{\large\textrm}
\lohead{\Aufgabe}
\rohead{Team-ID: \TeamId}
\cfoot*{\thepage{}/\pageref{LastPage}}

% Position des Titels
\usepackage{titling}
\setlength{\droptitle}{-1.0cm}

% Für mathematische Befehle und Symbole
\usepackage{amsmath}
\usepackage{amssymb}

% Für Bilder
\usepackage{graphicx}

% Für Algorithmen
\usepackage{algpseudocode}

% Für Quelltext
\usepackage{listings}
\usepackage{color}
\definecolor{mygreen}{rgb}{0,0.6,0}
\definecolor{mygray}{rgb}{0.5,0.5,0.5}
\definecolor{mymauve}{rgb}{0.58,0,0.82}
\lstset{
  keywordstyle=\color{blue},
  commentstyle=\color{mygreen},
  stringstyle=\color{mymauve},
  rulecolor=\color{black},
  basicstyle=\footnotesize\ttfamily,
  numberstyle=\tiny\color{mygray},
  captionpos=b, % sets the caption-position to bottom
  keepspaces=true, % keeps spaces in text
  numbers=left,
  numbersep=5pt,
  showspaces=false,
  showstringspaces=true,
  showtabs=false,
  stepnumber=2,
  tabsize=2,
  title=\lstname
}
\lstdefinelanguage{JavaScript}{ % JavaScript ist als einzige Sprache noch nicht vordefiniert
  keywords={break, case, catch, continue, debugger, default, delete, do, else, finally, for, function, if, in, instanceof, new, return, switch, this, throw, try, typeof, var, void, while, with},
  morecomment=[l]{//},
  morecomment=[s]{/*}{*/},
  morestring=[b]',
  morestring=[b]",
  sensitive=true
}

% Diese beiden Pakete müssen zuletzt geladen werden
%\usepackage{hyperref} % Anklickbare Links im Dokument
\usepackage{cleveref}

% Daten für die Titelseite
\title{\textbf{\Huge\Aufgabe}}
\author{\LARGE Team-ID: \LARGE \TeamId \\\\
	    \LARGE Team-Name: \LARGE \TeamName \\\\
	    \LARGE Bearbeiter/-innen dieser Aufgabe: \\ 
	    \LARGE \Namen\\\\}
\date{\LARGE\today}

\begin{document}

\maketitle
\tableofcontents

\vspace{0.5cm}

\textbf{Anleitung:} Tragen Sie oben in den Zeilen 8 bis 11 die Aufgabennummer, die Team-ID, den Team-Namen und alle Bearbeiter/-innen dieser Aufgabe mit Vor- und Nachnamen ein. Stellen Sie sicher, dass der Aufgabentitel angepasst wird (anstatt "`\LaTeX-Dokument"').

Verwenden Sie anschließend Ihre \LaTeX-Umgebung, um dieses Dokument zu kompilieren.

Die vorliegenden Texte enthalten Hinweise zur Einsendung. Diese sollten in Ihrer endgültigen Einsendung entfernt werden.

\section{Lösungsidee}

Das Ziel der Aufgabe ist es, einen Algorithmus zu entwerfen, der drei Streckenlängen ermittelt, die die größtmögliche Anzahl von Mitgliedern der Informatik-Gesundheitskasse (IGK) zur Teilnahme am Wandertag motivieren. Um dies zu erreichen, muss der Algorithmus eine Auswahl an Streckenlängen treffen, die den maximalen zusätzlichen Nutzen hinsichtlich der Teilnehmerabdeckung bietet.

Die Grundidee der Lösung besteht darin, die Streckenlängen systematisch auszuwählen, um die Anzahl der Mitglieder zu maximieren, die durch die gewählten Strecken abgedeckt werden. Hierzu wird folgender Ansatz verfolgt:

\subsection{Identifikation der relevanten Streckenlängen}

Zu Beginn werden alle möglichen Streckenlängen identifiziert, die von den Mitgliedern der IGK als akzeptabel angegeben wurden. Jede Strecke wird in einem Bereich von minimaler bis maximaler Länge beschrieben, den die Mitglieder bereit sind zu absolvieren.

\subsection{Priorisierung der Streckenlängen}

Jede Strecke wird bewertet, basierend auf der Anzahl der Mitglieder, die diese Strecke als akzeptabel angegeben haben. Das bedeutet, dass Strecken, die von vielen Mitgliedern als passend angesehen werden, eine höhere Priorität erhalten.

\subsection{Auswahl der optimalen Strecken}

Der Algorithmus wählt iterativ die Streckenlängen aus, die den größten zusätzlichen Nutzen bieten. Der zusätzliche Nutzen bezieht sich auf die Anzahl der zusätzlichen Mitglieder, die durch die Wahl einer bestimmten Strecke abgedeckt werden, ohne bereits von den vorher gewählten Strecken abgedeckt zu sein.

\subsection{Berücksichtigung der Überlappungen}

Bei der Auswahl wird darauf geachtet, dass die bereits durch ausgewählte Strecken abgedeckten Mitglieder nicht mehrfach gezählt werden. Die Wahl neuer Strecken erfolgt daher immer unter Berücksichtigung der bereits erreichten Abdeckung.

\subsection{Abschluss der Auswahl}

Der Auswahlprozess wird fortgesetzt, bis die festgelegte Anzahl von drei Streckenlängen erreicht ist. Diese drei Strecken sollten die größtmögliche Anzahl an Mitgliedern abdecken, basierend auf den vorab getroffenen Entscheidungen.

Zusammengefasst wird durch diesen Ansatz eine Auswahl an Streckenlängen getroffen, die optimal auf die Bedürfnisse der Mitglieder abgestimmt ist und die Teilnehmerzahl maximiert. Der Algorithmus zielt darauf ab, die beste Balance zwischen Anzahl der abgedeckten Mitglieder und der Anzahl der ausgewählten Strecken zu finden, um das Ziel der maximalen Teilnehmerbeteiligung zu erreichen.

\section{Umsetzung}

Der verwendete Algorithmus basiert auf einer Greedy-Methode, bei der in jedem Schritt die Strecke gewählt wird, die die größte Anzahl zusätzlicher Mitglieder abdeckt, die noch nicht durch die bereits gewählten Strecken abgedeckt sind. Dieser Prozess wird fortgesetzt, bis drei Streckenlängen ausgewählt sind. Die Wahl der Strecken erfolgt anhand der maximalen Überlappung der Mitgliederwünsche mit den bereits gewählten Strecken. Die Greedy-Strategie garantiert, dass in jedem Schritt die beste verfügbare Option gewählt wird, um eine möglichst hohe Abdeckung zu erreichen.

\section{Beispiele}

Zur Veranschaulichung der Funktionsweise des Algorithmus werden verschiedene Beispiele präsentiert, bei denen die Streckenlängen und die Anzahl der Teilnehmer getestet werden. Dabei werden sowohl Beispiele von der BwInf-Webseite als auch eigene Beispiele verwendet, um die Robustheit und Effektivität des Algorithmus zu überprüfen. Jedes Beispiel wird ausführlich beschrieben und die Ergebnisse des Algorithmus werden erläutert.

\section{Quellcode}

\lstinputlisting[language=Python,caption=Greedy-Algorithmus für die Streckenlängenberechnung]{wandertag.py}

\end{document}
